\documentclass[a4paper,10pt]{scrartcl}
%\documentclass[a4paper]{article}
\usepackage[T1]{fontenc}
\usepackage[utf8]{inputenc}
\usepackage{lmodern}
\usepackage[ngerman]{babel}
\usepackage{geometry}
\geometry{a4paper,left=20mm,right=20mm, top=0cm, bottom=1.5cm} 

\def\fach{Physik}


\def\einsichtStart{16.05.2012}
\def\einsichtEnde{22.05.2012}

\def\wahlvvDatum{22.05.2012}
\def\wahlvvUhr{14:30 Uhr}
\def\wahlvvSaal{5K}

\def\buero{Fachschaft Mathematik, 25.22.U1.25}
\def\wahlanfang{30.05.2012}
\def\wahlende{01.06.2012}
\def\ort{ist vor der MatNatFak Bibliothek}

\def\maxWahl{11}

\def\wahlleiter{Sergej Poplavski}
\def\wahlpersonzwei{Janine Haas}
\def\wahlpersondrei{Andreas Troll}

\def\briefwahlAntrag{23.05.2012}
\def\briefwahlEnde{01.06.2012, 14:15 Uhr}

\def\auszaehlung{01.06.2012, 14:15 Uhr}


\title{Wahlbekanntmachung}
\subtitle{für die Wahl zum Fachschaftsrat \fach \\ in der Zeit vom \wahlanfang ~- \wahlende}
\parindent 0pt
\parskip 10pt
\date{}
\begin{document}
\pagestyle{empty}
\thispagestyle{empty}
\renewcommand{\titlepagestyle}{empty}
\vspace{-0.5cm}
\maketitle
\vspace{-5\baselineskip}
Vom \wahlanfang ~bis zum \wahlende ~wird ein neuer Fachschaftsrat \fach ~gewählt. Wahlberechtigt und wählbar sind alle Studierenden der Fachschaft \fach.
Alle diese Studierenden sind im Wählerverzeichnis aufgeführt, welches vom \einsichtStart ~bis zum \einsichtEnde ~im Studierendensekretariat zur Einsichtnahme ausliegt. Einsprüche gegen die Richtigkeit des Wählerverzeichnises können beim Wahlleiter bis zum \einsichtEnde ~schriftlich erklärt werden. Über den Einspruch entscheidet dann der Wahlausschuss. Ist ein Studierender nicht im Wählerverzeichnis, aber er hat keinen Einspruch erhoben, so obliegt ihm der Nachweis seiner Wahlberechtigung, dafür ist ein gültige Immatrikalitionsbescheinigung und ein gültiger Lichtbildausweis ausreichend.

Stand der Wahlurne \ort.

Es ist zu jeder Zeit eine Wahl ohne Studierendenausweis möglich, ausgenommen sind Studierende, die nicht im Wählerverzeichnis aufgeführt werden.

Gewählt wird zu folgenden Zeiten:
\begin{itemize}
\item Mittwoch, den 30. Mai 11:15 Uhr – 14:15 Uhr
\item Donnerstag, den 31. Mai 09:15 Uhr – 13:15 Uhr
\item Freitag, den 01. Juni 12:15 Uhr – 14:15 Uhr 

\end{itemize}

Gewählt werden voraussichtlich bis zu \maxWahl ~Mitglieder (nach den Zahlen des Wintersemesters 2011/12).

\textbf{Wahlsystem}

Jeder Wahlberechtigte kann f"ur jeden Kandidaten eine positive (Ja) oder eine negative Stimme (Nein) abgeben. Desweiteren besteht die M"oglichkeit der Enthaltung (Kreuz bei Enthaltung oder kein Kreuz). Es sind also maximal so viele Kreuze zu setzen, wie Kandidaten zur Wahl stehen, jedoch auf keinen Kandidaten mehr als ein Kreuz.

Wurde bei min. einem Kandidaten mehr als eine Stimme abgegeben, so ist der \textbf{Stimmzettel} ung\"ueltig. Ist der Wille des W\"ahlers nicht eindeuitg erkennbar, oder enth"alt die Stimme handschriftliche Zus"atze, so ist die \textbf{Stimme} ung"ultig.

Gewählt sind die KandidatInnen, bei denen die Differenz der Positiv- und Negativstimmen größer oder gleich eins (>=1) ist. 
Ist die Zahl der gewählten KandidatInnen größer als die Zahl der zu vergebenden Sitze, so wird eine Reihung unter diesen KandidatInnen gemäß der erreichten Differenz vorgenommen. 
Bei Differenzgleichheit werden die KandidatInnen mit absolut weniger Negativstimmen vorgezogen. Bei identischer Anzahl an Negativstimmen entscheidet das Los über den Rang. 
Die Sitze werden den KandidatInnen in der Reihenfolge der von ihnen erreichten Differenz zugeteilt.


Die \textbf{Wahlvollversammlung} findet am \wahlvvDatum ~um \wahlvvUhr ~Uhr im Hörsaal \wahlvvSaal ~statt.

Wahlvorschläge können immer beim Wahlleiter (\wahlleiter ), insbesondere auf der Wahlvollversammlung bis zum Abschluss des Tagesordnungspunktes ``Nominierung, Vorstellung und Befragung der Kandidaten zur Wahl des Fachschaftsrates'' abgegeben werden. 

Bis zum \briefwahlAntrag ~können beim Wahlausschuss schriftlich Briefwahlunterlagen beantragt werden. Die Briefwahl\-stimme muss bis spätestens \briefwahlEnde beim Wahlleiter eingegangen sein.

Die Stimmen werden im Anschluss an die Wahl am \auszaehlung ~in der ~\buero \phantom{.} ausgez"ahlt.


Düsseldorf, den \today

\vspace{2\baselineskip}

\wahlleiter
\end{document}
