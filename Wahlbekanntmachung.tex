\documentclass[a4paper,10pt]{scrartcl}
%\documentclass[a4paper]{article}
\usepackage[T1]{fontenc}
\usepackage[utf8]{inputenc}
\usepackage{lmodern}
\usepackage[ngerman]{babel}
\usepackage{geometry}
\geometry{a4paper,left=20mm,right=20mm, top=0cm, bottom=2cm} 
\def\fach{Mathematik}
\def\wahlvvDatum{21.06.2010}
\def\wahlvvUhr{12:30}
\def\wahlvvSaal{5E}
\def\buero{Fachschaftsraum Mathematik, 25.22.U1.25}
\def\raum{25.22.U1.25}
\def\zeit{4. - 6. Juli 2011}
\def\einsichtStart{20. Juni}
\def\einsichtEnde{26. Juni}
\def\ort{ist vor der MatNatFak Bibliothek}
\def\maxWahl{9}
\def\wahlleiter{Evgeni Golov}
\def\wahlpersonzwei{Jessica Hub}
\def\wahlpersondrei{Mathis Fricke}
\def\briefwahlAntrag{27. Juni}
\def\briefwahlEnde{06. Juli 15:00 Uhr}
\def\auszaehlung{Mittwoch den 06. Juli ab 15 Uhr}
\title{Wahlbekanntmachung}
\subtitle{für die Wahl zum Fachschaftsrat \fach \\ in der Zeit vom \zeit}
\parindent 0pt
\parskip 10pt
\date{}
\begin{document}
\pagestyle{empty}
\thispagestyle{empty}
\renewcommand{\titlepagestyle}{empty}
\vspace{-0.5cm}
\maketitle
\vspace{-2cm}
Vom \zeit~wird ein neuer Fachschaftsrat \fach~gewählt. Wahlberechtigt und wählbar sind alle Studierenden der Fachschaft \fach.
Alle diese Studierenden sind im Wählerverzeichnis aufgeführt, welches vom \einsichtStart~bis zum \einsichtEnde~im Studierendensekretariat zur Einsichtnahme ausliegt. Einsprüche gegen die Richtigkeit des Wählerverzeichnises können beim Wahlleiter bis zum \einsichtEnde~schriftlich erklärt werden. Über den Einspruch entscheidet dann der Wahlausschuss. Ist ein Studierender nicht im Wählerverzeichnis, aber er hat keinen Einspruch erhoben, so obliegt ihm der Nachweis seiner Wahlberechtigung, dafür ist ein gültige Immatrikalitionsbescheinigung und ein gültiger Lichtbildausweis ausreichend.

Stand der Wahlurne \ort.

Es ist zu jeder Zeit eine Wahl ohne Studierendenausweis möglich, ausgenommen sind Studierende, die nicht im Wählerverzeichnis aufgeführt werden.

Gewählt wird zu folgenden Zeiten:
\begin{itemize}
\item Montag den 04. Juli 11:30 Uhr – 16:00 Uhr
\item Dienstag den 05. Juli 10:00 Uhr – 15:00 Uhr
\item Mittwoch den 06. Juli 11:30 Uhr – 14:00 Uhr 
\item Mittwoch den 06. Juli 14:00 Uhr – 15:00 Uhr im Wahlbüro (\buero) 
\end{itemize}

Gewählt werden können bis zu \maxWahl~Mitglieder.

\subsection*{Wahlsystem}
Jeder Wahlberechtigte hat so viele Stimmen, wie Kandidaten zur Wahl stehen, jedoch nicht mehr als neun, da dies die maximale Anzahl an Mitgliedern ist. Jedem Kandidaten darf maximal eine Stimme gegeben werden. Eine Stimmhäufung ist nicht möglich. 

Wahlzettel mit mehr Stimmen als Kandidaten, mehr als neun Stimmen, Stimmhäufung oder bei denen der Wille des Wählers nicht eindeutig erkennbar ist, werden als ungültig gewertet.

Hat ein Kandidat mehr als 50\% der abgegebenen, gültigen Stimmen, so gilt er als gewählt. Gibt es mehr als 9 gewählte Kandidaten, so entscheidet die Reihenfolge der Kandidaten gemäß der erreichten Stimmenzahl. Bei Stimmengleichheit entscheidet das Los. Die Sitze werden den Kandidaten in der Reihenfolge der von ihnen erreichten Stimmenzahl zugeteilt.

Die Wahlvollversammlung findet am \wahlvvDatum~um \wahlvvUhr~Uhr im Hörsaal \wahlvvSaal~statt.

Wahlvorschläge können bis zum Abschluss des Tagesordnungspunktes ``Nominierung, Vorstellung und Befragung der Kandidaten zur Wahl des Fachschaftsrates'' auf der Wahlvollversammlung abgegeben werden. 

Bis zum \briefwahlAntrag~können beim Wahlausschuss schriftlich Briefwahlunterlagen beantragt werden. Die Briefwahlstimme muss bis spätestens \briefwahlEnde~beim Wahlleiter eingegangen sein.

Wahlvorschläge können jederzeit an Mitglieder des Wahlausschusses gegeben werden. Sie sind mindestens während ihrer Öffnungszeiten im Raum der Fachschaft anzutreffen.

Ausgezählt wird am \auszaehlung~im \buero.

\vspace{1em}

Düsseldorf den \today

\vspace{1em}

\begin{center}
\begin{tabular}{lll}
\wahlleiter \hspace{2cm} & \wahlpersonzwei & \hspace{2cm} \wahlpersondrei \\
(Wahlleiter) & & 
\end{tabular}
\end{center}
\end{document}
